\documentclass[11pt,a4paper]{article}
\usepackage[utf8]{inputenc}
\usepackage[T1]{fontenc}
\usepackage{amsmath}
\usepackage{amsfonts}
\usepackage{amssymb}
\usepackage{graphicx}
\usepackage{listings}
\usepackage{xcolor}
\usepackage{hyperref}
\usepackage{geometry}
\usepackage{float}
\usepackage{enumitem}

\geometry{margin=1in}

% Code listing style
\lstset{
    language=C++,
    basicstyle=\ttfamily\small,
    keywordstyle=\color{blue}\bfseries,
    commentstyle=\color{green!60!black},
    stringstyle=\color{red},
    numbers=left,
    numberstyle=\tiny\color{gray},
    stepnumber=1,
    numbersep=5pt,
    backgroundcolor=\color{gray!10},
    frame=single,
    breaklines=true,
    breakatwhitespace=true,
    tabsize=2,
    showstringspaces=false
}

\title{MAV Formation Control System\\ROS2 Workspace Documentation}
\author{ROS2 Formation Control Package}
\date{\today}

\begin{document}

\maketitle

\section{Overview}

This ROS2 workspace implements a hierarchical formation control system for Micro Aerial Vehicles (MAVs) using ROS2 Humble. The system enables coordinated movement of multiple drones in a structured formation, following a virtual leader that traverses a predefined trajectory. The architecture supports a two-level hierarchy: \textit{leader} drones that follow the virtual leader, and \textit{follower} drones that follow their respective leaders.

\section{System Architecture}

The formation control system consists of three main ROS2 nodes that work together to achieve coordinated multi-agent flight:

\begin{enumerate}
    \item \textbf{Virtual Leader Trajectory Node}: Generates and publishes the trajectory of a virtual leader based on waypoints
    \item \textbf{Formation Controller Node}: Controls the position of leader drones around the virtual leader
    \item \textbf{Follower Controller Node}: Controls the position of follower drones around their respective leaders
\end{enumerate}

\subsection{Communication Flow}

The system follows a hierarchical communication pattern:
\begin{itemize}
    \item The \textit{virtual leader trajectory node} publishes the virtual leader's pose
    \item The \textit{formation controller node} subscribes to the virtual leader's pose and publishes poses for each leader drone
    \item The \textit{follower controller node} subscribes to each leader's pose and publishes poses for each follower drone
\end{itemize}

\section{Node Descriptions}

\subsection{Virtual Leader Trajectory Node}

\textbf{File}: \texttt{src/virtual\_leader\_trajectory\_node.cpp}

This node generates a smooth trajectory for a virtual leader by interpolating between waypoints defined in a configuration file. The virtual leader serves as the reference point for the entire formation.

\subsubsection{Functionality}
\begin{itemize}
    \item Loads waypoints from a text file (\texttt{config/waypoints.txt})
    \item Interpolates between waypoints at a configurable speed
    \item Implements smooth yaw (orientation) transitions with rate limiting
    \item Loops through waypoints continuously
    \item Publishes the virtual leader's pose and visualization markers
\end{itemize}

\subsubsection{Key Features}
\begin{itemize}
    \item \textbf{Waypoint Navigation}: Moves from waypoint to waypoint with configurable tolerance
    \item \textbf{Smooth Orientation}: Implements maximum yaw rate limiting for realistic orientation changes
    \item \textbf{Trajectory Interpolation}: Uses linear interpolation between waypoints based on trajectory speed
    \item \textbf{Visualization}: Publishes a green sphere marker representing the virtual leader in RViz2
\end{itemize}

\subsubsection{Published Topics}
\begin{itemize}
    \item \texttt{/virtual\_leader/pose} (geometry\_msgs/PoseStamped): Current pose of the virtual leader
    \item \texttt{/virtual\_leader/marker} (visualization\_msgs/Marker): Visualization marker (green sphere)
\end{itemize}

\subsubsection{Parameters}
\begin{itemize}
    \item \texttt{waypoint\_file}: Path to waypoint file (default: \texttt{config/waypoints.txt})
    \item \texttt{trajectory\_speed}: Speed of virtual leader in m/s (default: 1.0)
    \item \texttt{waypoint\_tolerance}: Distance tolerance to consider waypoint reached in meters (default: 0.1)
    \item \texttt{max\_yaw\_rate}: Maximum yaw rate in rad/s for smooth orientation changes (default: 1.0)
    \item \texttt{frame\_id}: TF frame ID (default: \texttt{map})
    \item \texttt{publish\_rate}: Publishing frequency in Hz (default: 50.0)
\end{itemize}

\subsection{Formation Controller Node}

\textbf{File}: \texttt{src/formation\_controller\_node.cpp}

This node maintains a circular formation of leader drones around the virtual leader. The leaders are arranged at equal angular intervals around the virtual leader at a specified radius.

\subsubsection{Functionality}
\begin{itemize}
    \item Subscribes to the virtual leader's pose
    \item Calculates desired positions for each leader drone based on formation geometry
    \item Rotates formation offsets based on the virtual leader's orientation
    \item Publishes poses for each leader drone
    \item Publishes visualization markers for all leaders
\end{itemize}

\subsubsection{Formation Geometry}
The leaders are arranged in a circle around the virtual leader:
\begin{equation}
\begin{aligned}
x_i &= x_{VL} + r_f \cos\left(\frac{2\pi i}{N} + \theta_0\right) \cos(\psi) - r_f \sin\left(\frac{2\pi i}{N} + \theta_0\right) \sin(\psi) \\
y_i &= y_{VL} + r_f \cos\left(\frac{2\pi i}{N} + \theta_0\right) \sin(\psi) + r_f \sin\left(\frac{2\pi i}{N} + \theta_0\right) \cos(\psi) \\
z_i &= z_{VL}
\end{aligned}
\end{equation}
where:
\begin{itemize}
    \item $(x_i, y_i, z_i)$ is the position of leader $i$
    \item $(x_{VL}, y_{VL}, z_{VL})$ is the virtual leader position
    \item $r_f$ is the formation radius
    \item $N$ is the number of leaders
    \item $\psi$ is the virtual leader's yaw angle
    \item $\theta_0$ is the formation angle offset
\end{itemize}

\subsubsection{Published Topics}
\begin{itemize}
    \item \texttt{/leader\_<i>/pose} (geometry\_msgs/PoseStamped): Pose of leader $i$
    \item \texttt{/formation\_leaders/markers} (visualization\_msgs/MarkerArray): Visualization markers for all leaders (opaque colored spheres)
\end{itemize}

\subsubsection{Parameters}
\begin{itemize}
    \item \texttt{num\_leaders}: Number of leader drones (default: 2)
    \item \texttt{formation\_radius}: Radius of formation around virtual leader in meters (default: 2.0)
    \item \texttt{formation\_angle\_offset}: Angular offset for formation arrangement in radians (default: 0.0)
    \item \texttt{frame\_id}: TF frame ID (default: \texttt{map})
    \item \texttt{publish\_rate}: Publishing frequency in Hz (default: 50.0)
\end{itemize}

\subsection{Follower Controller Node}

\textbf{File}: \texttt{src/follower\_controller\_node.cpp}

This node maintains a circular formation of follower drones around each leader. Each leader has its own group of followers arranged in a circle.

\subsubsection{Functionality}
\begin{itemize}
    \item Subscribes to each leader's pose
    \item Calculates desired positions for each follower based on follower geometry
    \item Rotates follower offsets based on their leader's orientation
    \item Publishes poses for each follower drone
    \item Publishes visualization markers for all followers (transparent, matching leader color)
\end{itemize}

\subsubsection{Follower Geometry}
Similar to the leader formation, followers are arranged in a circle around their leader:
\begin{equation}
\begin{aligned}
x_{ij} &= x_{L_i} + r_{fol} \cos\left(\frac{2\pi j}{M} + \phi_0\right) \cos(\psi_i) - r_{fol} \sin\left(\frac{2\pi j}{M} + \phi_0\right) \sin(\psi_i) \\
y_{ij} &= y_{L_i} + r_{fol} \cos\left(\frac{2\pi j}{M} + \phi_0\right) \sin(\psi_i) + r_{fol} \sin\left(\frac{2\pi j}{M} + \phi_0\right) \cos(\psi_i) \\
z_{ij} &= z_{L_i}
\end{aligned}
\end{equation}
where:
\begin{itemize}
    \item $(x_{ij}, y_{ij}, z_{ij})$ is the position of follower $j$ of leader $i$
    \item $(x_{L_i}, y_{L_i}, z_{L_i})$ is the position of leader $i$
    \item $r_{fol}$ is the follower radius
    \item $M$ is the number of followers per leader
    \item $\psi_i$ is leader $i$'s yaw angle
    \item $\phi_0$ is the follower angle offset
\end{itemize}

\subsubsection{Published Topics}
\begin{itemize}
    \item \texttt{/follower\_<leader\_id>\_<follower\_id>/pose} (geometry\_msgs/PoseStamped): Pose of each follower
    \item \texttt{/formation\_followers/markers} (visualization\_msgs/MarkerArray): Visualization markers for all followers (transparent colored spheres)
\end{itemize}

\subsubsection{Parameters}
\begin{itemize}
    \item \texttt{num\_leaders}: Number of leader drones (must match formation controller) (default: 2)
    \item \texttt{followers\_per\_leader}: Number of followers per leader (default: 2)
    \item \texttt{follower\_radius}: Radius of followers around their leader in meters (default: 1.0)
    \item \texttt{follower\_angle\_offset}: Angular offset for follower arrangement in radians (default: 0.0)
    \item \texttt{frame\_id}: TF frame ID (default: \texttt{map})
    \item \texttt{publish\_rate}: Publishing frequency in Hz (default: 50.0)
\end{itemize}

\section{Visualization}

The system provides visualization markers for RViz2:

\begin{itemize}
    \item \textbf{Virtual Leader}: Green sphere (opaque)
    \item \textbf{Leader Drones}: Colored spheres (opaque) - Red, Blue, Yellow, Magenta
    \item \textbf{Follower Drones}: Colored spheres (transparent, alpha=0.5) - Same color as their leader
\end{itemize}

\section{Configuration}

\subsection{Waypoint File Format}

Waypoints are defined in \texttt{config/waypoints.txt} with the format:
\begin{lstlisting}[language=bash]
# Comments start with #
x y z
\end{lstlisting}

Example:
\begin{lstlisting}[language=bash]
0.0 0.0 1.0
2.0 0.0 1.0
2.0 2.0 1.0
0.0 2.0 1.0
\end{lstlisting}

\subsection{Parameter Configuration}

All parameters are configured in \texttt{config/params.yaml}. The system supports:
\begin{itemize}
    \item Configurable number of leaders and followers
    \item Adjustable formation and follower radii
    \item Configurable trajectory speed
    \item Adjustable publishing rates
    \item Customizable frame IDs
\end{itemize}

\section{Launch System}

The system is launched using ROS2 launch files:

\begin{itemize}
    \item \texttt{formation\_control.launch.py}: Launches all three nodes with parameters from \texttt{params.yaml}
    \item \texttt{formation\_control\_with\_rviz.launch.py}: Launches the system with RViz2 visualization
    \item \texttt{rviz.launch.py}: Launches only RViz2 with pre-configured visualization
\end{itemize}

\section{Usage}

\subsection{Building the Package}

\begin{lstlisting}[language=bash]
cd ~/robotarium/ros2_smc_ws
colcon build --packages-select mav_formation_control
source install/setup.bash
\end{lstlisting}

\subsection{Running the System}

\begin{lstlisting}[language=bash]
# Launch complete system
ros2 launch mav_formation_control formation_control.launch.py

# Launch with RViz2
ros2 launch mav_formation_control formation_control_with_rviz.launch.py
\end{lstlisting}

\subsection{Parameter Override Example}

\begin{lstlisting}[language=bash]
ros2 launch mav_formation_control formation_control.launch.py \
  num_leaders:=4 \
  followers_per_leader:=3 \
  trajectory_speed:=1.5 \
  formation_radius:=3.0
\end{lstlisting}

\section{Technical Details}

\subsection{Dependencies}

The package depends on:
\begin{itemize}
    \item \texttt{rclcpp}: ROS2 C++ client library
    \item \texttt{geometry\_msgs}: Geometric message types
    \item \texttt{visualization\_msgs}: Visualization markers
    \item \texttt{tf2}, \texttt{tf2\_ros}, \texttt{tf2\_geometry\_msgs}: Transform library
    \item \texttt{ament\_index\_cpp}: Package resource access
\end{itemize}

\subsection{Coordinate Frames}

The system operates in the \texttt{map} frame by default. All poses are published with respect to this frame, and the formation geometry is calculated in this coordinate system.

\section{Conclusion}

This ROS2 workspace provides a complete hierarchical formation control system for MAVs. The three-node architecture enables scalable formation control with configurable geometry, supporting various formation patterns through parameter adjustment. The system is designed for simulation and can be extended for real-world deployment with appropriate hardware interfaces.

\end{document}

